 

\section{Integraci\'on por sustituci\'on}
 En la anterior secci\'on, trabajamos reglas de integraci\'on  relacionadas con las reglas de diferenciaci\'on b\'asica. Ahora, presentamos un m\'etodo de integraci\'on llamado m\'etodo de sustituci\'on o cambio de variable, que est\'a relacionado con la regla de la cadenas. A partir de esta regla trataremos de buscar un m\'etodo eficaz para resolver integrales de funciones compuestas.

 Sea $F:I\rightarrow \R$ una primitiva de la funci\'on $f:I\rightarrow \R$ en el intevalo  $I$ y $g:I\rightarrow \R$ una funci\'on derivable tal que $\mbox{Im}(g) \subset I$. Usuando la regla de la cadena, tenemos
$$(F(g(x)))^{\prime}=F^{\prime}(g(x))g^{\prime}(x)=f(g(x))g^{\prime}(x).$$
Luego, $F(g(x))$ es una primitiva de $f(g(x))g^{\prime}(x)$. Por tanto,
\begin{align} \label{sustitucion} 
 \int f(g(x))g^{\prime}(x) dx= F(g(x))+ C. 
\end{align}
%\noindent suponiendo que $F^{\prime}(x)=f(x)$. Es decir, $F(x)$ es antiderivada de $f(x)$. De lo anterior se establece la  técnica de integración, llamada  sustituci\'on, que permite calcular integrales de la forma
%$$\int f(g(x))\cdot g^{\prime}(x)dx.$$
Realizando la sustituci\'on, 
$$ u=g(x) \Rightarrow du=g^{\prime}(x)\ dx, $$
tenemos que la expresi\'on \eqref{sustitucion}, se transforma en la siguiente 
\begin{align*}
 \int f(g(x))  g^{\prime}(x)dx= \int f(u)du = F(u) + C =F(g(x))+C.
\end{align*}
Como consecuencia de lo anterior, obtenemos el siguiente teorema, 
\begin{teorema}{Sustituci\'on}{AFC}\label{teo-regla-cadena-int}
Sea $F:I\rightarrow \R$ una primitiva de la funci\'on $f:I\rightarrow \R$ en el intevalo  $I$ y $g:I\rightarrow \R$ una funci\'on derivable tal que $\mbox{Im}(g) \subset I$. Entonces $$\int f(g(x))g'(x)dx = F(g(x))+C.$$
\end{teorema}
%%%%%%%%%%%%%%%%%%%%%%%%%%%%%%%%%%%%%%%%%%%%%%%%%%%%%%%%%%%%%%%%%%%%%%%%%%%%%%%%%%%%%%%%%%%%%%%%%%%%%%%%%%%%%%%%%%%%%%%%%%%%%%%%%%%%%%%%%%%%%%%%%%%%%%%%%%%%%%%%%%%%%%%%%%%%%%%%%%%
\begin{Ejemplo} $\displaystyle\int \sin^{3}(x)\cos(x)\,dx$
\end{Ejemplo}
\begin{proof}
%Observemos que la integral no se calcula a partir de las reglas b\'asicas de integraci\'on. Adem\'as,
%$$\displaystyle\int \cos(x)\sin^{3}(x)\,dx \neq \displaystyle\int \cos(x)\,dx \cdot \displaystyle\int  \sin^{3}(x)\,dx. $$
%Sin embargo, en el integrando se encuentra $\sin(x)$ y su derivada, $\cos(x)$. Esto \'ultimo nos permite reconocer el patr\'on tomando 
Si  $\red{u=\sin(x)}$, entonces $\red{du=\cos(x)\,dx }.$ Al reemplazar, resulta una integral inmediata,
\begin{equation*}
\int \sin^{3}(x)\cos(x)\,dx=\int u^{3}\,du=\frac{u^4}{4}+C.
\end{equation*}
Una vez resulta, retomamos la variable inicial  $u=\sin(x)$
$$\displaystyle\int \sin^{3}(x)\cos(x)\,dx =\frac{1}{4}\sin^{4} (x) +C$$
\end{proof}


%%%%%%%%%%%%%%%%%%%%%%%%%
%\newpage

\begin{Ejemplo} Calcular $\displaystyle\int x\sqrt{1+x^{2}}\,dx$
 \end{Ejemplo}
 \begin{proof}

Sea $\red{u=1+x^{2}},$ entonces $\red{du=2x\,dx}$. A diferencia del ejemplo anterior el diferencial no aparece de forma inmediata y es necesario  agregarle factores constantes al integrando que no alteren la funci\'on. Por ejemplo,
$$\displaystyle\int x\sqrt{1+x^{2}}\,dx=\frac{2}{2}\displaystyle\int  x\sqrt{1+x^{2}}\,dx=\frac{1}{2}\displaystyle\int  2x\sqrt{1-x^{2}}\,dx $$
Lo anterior se suele hacer despejando  el diferencial
$$du=2x\,dx \  \Rightarrow \ \frac{1}{2} du=x\,dx $$
Sustituyendo en la integral tenemos
\begin{align*}
\displaystyle\int x\sqrt{1+x^{2}}\,dx &=\frac{1}{2}
\int u^{1/2}\,du=\frac{1}{3}u^{3/2}+C\\
&=\frac{1}{3}(1+x^{2})^{3/2}+C
\end{align*}
\end{proof}


%%%%%%%%%%%%%%%%%%%%%%%%%

%\begin{Ejemplo} Calcula $\displaystyle\int(5+3x^3)^4 x^2 dx$
%\end{Ejemplo}
%\begin{proof}

%\end{proof}
% Sea $\red{  u = 5+3x^3 }$ entonces   $\red{ du  =  9x^2dx}$ y $\red{\dfrac{du}{9}=x^2dx}$
%  \begin{eqnarray*}
%    \displaystyle\int(5+3x^3)^4 x^2 dx = \frac{1}{9}\int u^4 du=\frac{u^5}{45}+C=\frac{(5+3x^3)^5}{45}+C
%     \end{eqnarray*}

%%%%%%%%%%%%%%%%%%%%%%%%%%
  
%\begin{Ejemplo} Calcula $\displaystyle\int x\cos (x^2) dx $
%\end{Ejemplo}
%\solucion \\
%Sea $ \red{u  =  x^2}$ entonces \quad  $\red{du  =  2xdx} $
%  \begin{eqnarray*}
%   \displaystyle\int x\cos (x^2) dx = \frac{1}{2}\int\cos (u)\;d\;u\; =\frac{1}{2} \sen u+C=\frac{1}{2}\sen (x^2)+C\\
%      \end{eqnarray*}

%%%%%%%%%%%%%%%%%%%%%%%%%


 \begin{Ejemplo} Calcula $\displaystyle\int \left[ (2x-1)e^{x^2-x}+2\pi \sen (2\pi x) \right] dx $
\end{Ejemplo}
\begin{proof}

Aplicando las propiedad
$$\displaystyle\int \left[ (2x-1)e^{x^2-x}+2\pi \sen (2\pi x) \right] dx =\displaystyle\int (2x-1)e^{x^2-x}  dx + \displaystyle\int  2\pi \sen (2\pi x)   dx $$
Identificamos que cada integral tiene una sustituci\'on diferente. Sean
\begin{center}
$ \red{u  =  x^2-x}$  \quad  $\red{du  = ( 2x-1)dx} $ \ y  \ $ \red{v  =  2\pi x}$  \quad  $\red{dv  =  2\pi dx} $
\end{center}
\vspace*{-0.1cm}
\begin{align*}
 \displaystyle\int \left[ (2x-1)e^{x^2-x}+2\pi \sen (2\pi x) \right]\ dx &= \int e^{u}du + \int \sen (v)dv  \\
 &= e^{x^2-x}- \cos (2\pi x)+ C
\end{align*}
\end{proof}

%%%%%%%%%%%%%%%%%%%%%%%%%

\textcolor{red!50!black}{\Large \bf Sustituir y despejar} \\

\noindent En los siguientes ejercicios no basta con identificar la funci\'on y su derivada, es necesario realizar un despeje para poder sustituir el integrando.

\newpage

\begin{Ejemplo} Hallar $\displaystyle\int x^5 \sqrt{x^2+4}\,dx$  \end{Ejemplo}
 \begin{proof}
Al tomar $\red{u=x^{2}+4}$,  se obtiene   $\red{\frac{1}{2}\,du=x\,dx}$. En el integrando no aparece explicitamente el diferencial pero  al aplicar las  propiedades de la potenciaci\'on lo obtenemos,
$$x^5 \ dx=x^4\red{ x\ dx}$$
Ahora, nuestra integral tiene la siguiente forma
 $$\displaystyle\int x^5 \sqrt{x^2+4}\,dx= \frac{1}{2}\displaystyle\int x^{4}  \sqrt{u}\,du$$
Inicialmente la sustituci\'on no permite cambiar totalmente la variable en el integrand, pero al despejar x en términos de u y elevar al cuadrado superamos el himpase.
\begin{center}
 $\red{u=x^{2}+4}$ entonces $\displaystyle
\red{u-4=x^2 }$ \  y \ $\red{(u-4)^2 =x^4} $.\end{center}

Por lo tanto,
$$\int  x^5 \sqrt{x^2+4}  \,dx= \frac{1}{2}\int  (u-4)^2 u^{1/2} \,dx $$
Integrando y retomando la variable inicial resulta
\begin{align*}
 \frac{1}{2}\int  (u-4)^2 u^{1/2} \,dx &=\frac{1}{2}\int (u^2-2u+16)u^{1/2} \ du \\
& = \frac{1}{2}\int (u^{5/2}-2u^{3/2}+16u^{1/2})\ du\\
& =\frac{1}{7}u^{7/2}-\frac{2}{5}u^{5/2}+\frac{16}{3}u^{3/2}+C \\
&=\frac{1}{7}(x^{2}+4)^{7/2}-\frac{2}{5}(x^{2}+4)^{5/2}+\frac{16}{3}(x^{2}+4)^{3/2}+C
\end{align*}
\end{proof}

%%%%%%%%%%%%%%%%%%%%%%%%%


%\begin{Ejemplo} Hallar $ \displaystyle\int \dfrac{4lnx}{x \sqrt{lnx +4}} \,dx$ \end{Ejemplo}
% \solucion \\
%Sea $\red{u=lnx+4}$ y $\red{du=\frac{1}{x}\,dx}$. Para obtener el numerador despejamos; $\displaystyle\red{u-4=lnx }$.
%
%\begin{align*}
%\displaystyle\int \frac{4lnx}{x \sqrt{lnx +4}} \,dx
% =&   4\int\frac{u+4}{\sqrt{u}}   \,du
% =  4\int u^{1/2}+4u^{-1/2} dx \\
% = &\frac{8}{3} u^{3/2} + 32 u^{1/2} +C
% = \frac{8}{3} (lnx+4)^{3/2} + 32 (lnx+4)^{1/2}+ C
%\end{align*}

%%%%%%%%%%%%%%%%%%%%%%%%%
   

%\begin{Ejemplo} Hallar $\displaystyle\int\frac{e^{(5 \ln(x)-x)} \sqrt{x^2+4}}{e^{x}}\,dx$  \end{Ejemplo}
% \solucion  \\
%Simplificando obtenemos,
%\begin{eqnarray*}
%\displaystyle\int\frac{e^{(5 \ln(x)+x)} \sqrt{x^2+4}}{e^{x}}\,dx &=& \displaystyle\int\frac{e^{5 \ln(x)}e^{x} \sqrt{x^2+4}}{e^{x}}\,dx= \displaystyle\int x^5  \sqrt{x^2+4}  \,dx
%\\&=& \frac{1}{7}(x^{2}+4)^{7/2}-\frac{2}{5}(x^{2}+4)^{5/2}+\frac{16}{3}(x^{2}+4)^{3/2}+C
%\end{eqnarray*}

%%%%%%%%%%%%%%%%%%%%%%%%%
 

%\begin{Ejemplo} Calcula $\displaystyle \int\frac{x\, dx}{\sqrt{-2x-x^2}} dx $
%\end{Ejemplo}
%Inicialmente reescribimos el integrando sumando y retando uno en el denominador
%\begin{align*}
% \displaystyle\int\frac{x\, dx}{\sqrt{-2x-x^2}} &=  \int\frac{x\, dx}{\sqrt{-(2x+x^2)}} =  \int\frac{x\, dx}{\sqrt{1-(x^2+2x+1)}}\\
%    &=  \int\frac{x\, dx}{\sqrt{1-(x+1)^2}}
%\end{align*}
%Ahora, aplicamos la sustituci\'on   $\red{u=x+1}$ y $\red{du=dx}$, y despejando $\red{x=u-1}$  obtenemos,\\
%\begin{align*}
%\int\frac{x\, dx}{\sqrt{1-(x+1)^2}} &=\int\frac{u\, du}{\sqrt{1-u^2}} - \int\frac{du}{\sqrt{1-u^2}} \\
%&= - \sqrt{1-u^2} -  \arcsin(u) + C\\
%& = - \sqrt{1-u^2} - \arcsin(x+1) + C
%\end{align*}

%%%%%%%%%%%%%%%%%%%%%%%%%
%%%%%%%%%%%%%%%%%%%%%%%%%
%%%%%%%%%%%%%%%%%%%%%%%%%
%%%%%%%%%%%%%%%%%%%%%%%%%

%\begin{Ejemplo} Calcula $\displaystyle\int\frac{x^3}{\sqrt{1-2x^2}}dx $ \end{Ejemplo}
%\solucion  \\
% Tomando $\red{u^2=1-2x^2}$ \ obtenemos que \ $ \red{ -\dfrac{1}{2}udu=x\;dx}$. Entonces
%\begin{eqnarray*}
% \displaystyle\int\frac{x^3}{\sqrt{1-2x^2}} \ dx  &=&  \displaystyle\int\frac{x^2 \ (x \ dx )}{\sqrt{1-2x^2}}=\int \frac{ \left(\frac{1-u^{2} }{2}\right)\left(-\frac{1}{2}u \right) }{u}\ du \\
%&=& -\frac{1}{4}\displaystyle\int\frac{(1-u^2)u}{u}\;d\;u=-\frac{1}{4}\int(1-u^2)du \\
%&=& -\frac{u}{4}+\frac{u^{3}}{12}+C=-\frac{\sqrt{1-2x^2}}{4}+\frac{(1-2x^2)^{3/2}}{12}+C
% \end{eqnarray*}

%%%%%%%%%%%%%%%%%%%%%%%%%
%%%%%%%%%%%%%%%%%%%%%%%%%
%%%%%%%%%%%%%%%%%%%%%%%%%
%%%%%%%%%%%%%%%%%%%%%%%%%


 \textcolor{red!50!black}{\Large \bf Doble sustituci\'on } \\

En algunos casos se debe realizar dos sustituciones

\begin{Ejemplo}\label{eje212} Calcula $\displaystyle\int\frac{x\;dx}{(x+1)-\sqrt{x+1}}$  \end{Ejemplo}
 \begin{proof}

Sea $\red{u=x+1}$, entonces  $\red{du=dx}$.

$$\displaystyle\int\frac{x\;dx}{(x+1)-\sqrt{x+1}}=\displaystyle\int\frac{u-1\;du}{u-\sqrt{u}}=\int\frac{(u-1)du}{\sqrt{u}(\sqrt{u}-1)}$$
Observando que la sustituci\'on no fue suficiente para calcular la integrar, se plantea una nueva. \\

Sea $\red{z=\sqrt{u}}$. Entonces  $\red{dz=\dfrac{1}{2\sqrt{u}}\,du}$.
\begin{align*}
\displaystyle\int\frac{(u-1)du}{\sqrt{u}(\sqrt{u}-1)}&=2\displaystyle\int \frac{z^2-1}{z-1}dz \\
&= 2\displaystyle\int ( z +1 ) dz \\
&= z^2+2z+C \\
&=(\sqrt{u})^2+2\sqrt{u}+C \\
&=x+2\sqrt{x+1}+C \\
\end{align*}
\end{proof}


%%%%%%%%%%%%%%%%%%%%%%%%%



%\begin{Ejemplo}  Calcula $\displaystyle\int\frac{x\  dx}{\sqrt{1+x^2+\sqrt{(1+x^2)^3}}}dx$ \end{Ejemplo}
%\begin{proof}
%
%Sea $\red{u=1+x^2}$ y $\red{du=2xdx}$.
% $$\displaystyle\int\frac{xdx}{\sqrt{1+x^2+\sqrt{(1+x^2)^3}}}  =  \frac{1}{2}\int\frac{du}{\sqrt{u+u^{3/2}}}$$
%Realizando, nuevamente,  sustituci\'on con  $\red{u=z^2}$ y $\red{du=2z\ dz}$ obtenemos
%\begin{align*}
%\frac{1}{2}\displaystyle\int\frac{du}{\sqrt{u+u^{3/2}}}=& \frac{1}{2} \displaystyle\int\frac{2z\;dz}{\sqrt{z^2+z^3}}=\displaystyle\int\frac{z\;dz}{z\sqrt{1+z}}=\displaystyle\int\frac{dz}{(1+z)^{1/2}}\\
%\end{align*}
%Tomando a $w=1+z$ y $dw=dz$, resulta
%$$\displaystyle\int\frac{dz}{(1+z)^{1/2}}=\displaystyle\int w^{-1/2} dw=2w^{1/2}+C$$
% Por lo tanto,
% $$\displaystyle\int\frac{xdx}{\sqrt{1+x^2+\sqrt{(1+x^2)^3}}}dx = 2(1+\sqrt{1+x^2})^{1/2}+C $$
%
%\end{proof}
%
%
%%%%%%%%%%%%%%%%%%%%%%%%%%
% 
%
%\newpage

\textcolor{red!50!black}{\Large \bf Funciones Exponenciales y Logar\'itmicas}\\

\noindent A partir de la t\'ecnica de sustituci\'on las reglas b\'asicas pueden generalizarse. Por ejemplo,

\begin{table}[h]
\begin{tcolorbox}[boxrule=0.2pt, enhanced,sharp corners,   width=12cm,colframe=blue!8!black,colback=blue!5!white,drop lifted shadow=blue ]
\begin{multicols}{2}
\begin{enumerate}

  \item $\displaystyle \int \frac{f^{\prime}(x)}{f(x)} \,dx= \ln |f(x)|+C$
  \item $\displaystyle \int e^{f(x)}f^{\prime}(x)\,dx=e^{f(x)}+C$
  \columnbreak
  \item $\displaystyle \int a^{f(x)}f^{\prime}(x)\,dx=\frac{a^{f(x)}}{\ln(a)}+C.$
\end{enumerate}
\end{multicols}
\end{tcolorbox}
\end{table}

%\begin{observacion}{Propiedades del logaritmo}{}
%Si $a$ y $b$ son n\'umeros positivos y $n$ es racional, se satisfacen las siguientes propiedades
%\begin{enumerate}
%\item $\ln(1)=0$
%\item $\ln(ab)=\ln(a)+\ln(b)$
%\item $\ln(a^n)=n\ln(a)$
%\item $\ln \left(\frac{a}{b}\right)=\ln(a)-\ln(b)$
%\end{enumerate}
%\end{observacion}

%%%%%%%%%%%%%%%%%%%%%%%%%

\begin{Ejemplo} Hallar $\displaystyle\int \frac{1}{4x-1} \,dx$ 
\end{Ejemplo}
\begin{proof}
Haciendo $\red{u=4x-1}$ tenemos que  $\red{du=4\,dx}$

 \begin{align*}
 \int \frac{1}{4x-1} \,dx & = \frac{1}{4} \int \frac{1}{u} \,du  = \frac{1}{4}
 \ln|u|+C  =\frac{1}{4}\ln|4x-1|+C
 \end{align*}
\end{proof}

%%%%%%%%%%%%%%%%%%%%%%%%%


 \begin{Ejemplo} Calcular  $\displaystyle\int{\dfrac{1}{x\ln x}}\, dx$  \end{Ejemplo}
\begin{proof}
 Haciendo  $\red{u=\ln x}$,   entonces \ \ $\red{du=\frac{1}{x}dx}$,   de modo que

\begin{equation*}
\int \frac{dx}{x\ln x} =\int \frac{1}{\ln x}\frac{1}{x}\,dx   = \int \frac{1}{u}\,du
 = \ln |u|+ C,
\end{equation*}
 y recuperando la variable nos queda que
\begin{equation*}
\int \frac{1}{x\ln x}\,dx= \ln |\ln x| + C.
\end{equation*}
\end{proof}

%%%%%%%%%%%%%%%%%%%%%%%%%
 
% \begin{Ejemplo} Calcular  $\displaystyle\int{\dfrac{1}{x (\ln x)^{3}}}\, dx$  \end{Ejemplo}
% \solucion\\
%Sea $\red{u=\ln x}$ y  $\red{du=\frac{1}{x}\ dx}$
%
%\begin{equation*}
%\displaystyle\int{\dfrac{1}{x (\ln x)^{3}}}\, dx  =  \displaystyle\int\dfrac{du}{ u^{3}} =\displaystyle\int u^{-3}du= \frac{1}{2u^{2}}+C
%\end{equation*}



%%%%%%%%%%%%%%%%%%%%%

 \begin{Ejemplo} Encontrar $\displaystyle\int e^{3x+1}\,dx$ 
 \end{Ejemplo}
\begin{proof}
 Sea $\red{u=3x+1}$  entonces $ \red{du=3\,dx }$. Despejando \ \ $\red{\frac{du}{3}=dx} $
\begin{align*}
 \int e^{3x+1}\,dx & =\int e^{u}\left(\frac{du}{3}\right)  =\frac{1}{3}\int
e^{u}\,du  =\frac{1}{3}e^{u}+C =\frac{1}{3}e^{3x+1}+C
\end{align*}
\end{proof}

%%%%%%%%%%%%%%%%%%%%%

\begin{Ejemplo} Encontrar $\displaystyle\int 5xe^{-x^2}\,dx$  
\end{Ejemplo}
\begin{proof}
 Sea $\red{u=-x^{2}}$   entonces  $ \red{du=-2x\,dx}$. Despejando $ \red{-\frac{du}{2}=x\,dx}$
\begin{align*}
 \int 5xe^{-x^2}\,dx  &=5\int e^{-x^2}(x\,dx)=5\int e^{u}\left(-\frac{du}{2}\right)  \\ &=-\frac{5}{2}\int
e^{u}\,du =-\frac{5}{2}e^{u}+C  =-\frac{5}{2}e^{-x^2}+C
\end{align*}
\end{proof}

%\begin{Ejemplo} Calcular $\displaystyle \int 3^{x\ln(x)}(\ln(x)+1)\,dx$ \end{Ejemplo}
% \solucion\\
% Sea $\red{u=x\ln(x)}$   entonces  $ \red{du=(\ln(x)+1)\,dx} $.
%\begin{align*}
% \int 3^{x\ln(x)}(\ln(x)+1)\,dx =\int
%3^{u}\,du  =\frac{3^{u}}{\ln(3)}+C  =\frac{3^{x\ln(x)}}{\ln(3)}+C
%\end{align*}


%%%%%%%%%%%%%%%%%%%%%%%%%%%
%%%%%%%%%%%%%%%%%%%%%%%%%%%
%%%%%%%%%%%%%%%%%%%%%%%%%%%
%%%%%%%%%%%%%%%%%%%%%%%%%%%

\noindent \textcolor{red!50!black}{\Large \bf   Funciones trigonom\'etricas} \\

\noindent
En en la Secci\'on. 1.1 estudiamos las integrales de funciones trigonom\'etricas a partir de la relaci\'on con la
derivada. Por medio de la integral por sustituci\'on, se puede completar el conjunto de reglas
b\'asicas de integraci\'on   trigonom\'etricas b\'asicas. Por ejemplo la integral de $\tan(u)$, $\cot(u)$ o
$\csc(u)$ \\

%\begin{Ejemplo}\large
%Calcular $\displaystyle \int \tan(x)\,dx$
%\end{Ejemplo}
%
%\solucion Usando la identidad trigonom\'etrica se tiene que
%\begin{equation*}
%\int \tan(x)\,dx =\int \frac{\sen(x)}{\cos(x)}\,dx.
%\end{equation*}
%
%\noindent Haciendo $\red{u= \cos(x)}$, tenemos que $\red{du= -\sen(x)\,dx}$, por lo que
%
%\begin{equation*}
% \int \tan(x)\,dx =\int
%\frac{\sen(x)}{\cos(x)}\,dx  =-\int\frac{du}{u} =-\ln \left| u \right| +C
%=- \ln\left| \cos(x) \right| +C.
%\end{equation*}

\begin{Ejemplo}
Hallar la integral $\displaystyle\int\cot x\,dx$
\end{Ejemplo}
\begin{proof}
Usando la identidad trigonom\'etrica de $\cot(x)$ tenemos que

\begin{equation*}
\int \cot(x)\,dx =\int\frac{\cos(x)}{\sen(x)}\,dx.
\end{equation*}

\noindent Sea $\red{u=\sen(x)}$ entonces $\red{du=\cos(x)\,dx}$, y sustituyendo nos queda

\begin{equation*}
\int \cot(x)\,dx =\int\frac{\sen(x)}{\cos(x)}\,dx   =\int
\frac{du}{u}  =\ln|u|+C  =\ln\left|\sen(x)\right|+C.
\end{equation*}

%\noindent Como no se hizo cambio en los l\'imites de integraci\'on se sigue
%
%\begin{align*}
%\int_{\pi/4}^{\pi/2}\cot(x)\,dx& =\ln\left|\sen(x)\right|\Big]_{\pi/4}^{\pi/2}
%=\ln\left|\sen\left(\pi/2\right)\right|-\ln\left|\sen\left(\pi/4\right)\right| \\
%& =\ln(1)-\ln\left(\sqrt{2}/2\right) =-\ln\left(\sqrt{2}/2\right).
%\end{align*}
\end{proof}

%\newpage

\begin{Ejemplo}
Hallar $\displaystyle\int \sec(x) \,dx$
\end{Ejemplo}

\begin{proof}
 Multiplicamos y dividimos en el integrando por $\tan(x)+\sec(x)$ entonces
\begin{align*}
\int \sec(x) \,dx = \int \sec (x)  \frac{\tan(x)+\sec(x)}{\tan(x)+\sec(x)}\,dx = \int
\frac{\sec(x)\tan(x)+\sec^2 (x)}{\tan(x)+\sec(x)}\,dx.
\end{align*}

\noindent Ahora bien, sea $\red{u=\tan(x)+\sec(x)}$ por lo que $\red{du= (\sec(x)\tan(x)+\sec^2(x))\,dx}$
\begin{align*}
\int\frac{\sec(x)\tan(x)+\sec^2(x)}{\tan(x)+\sec(x)}\,dx=\int\frac{du}{u}=\ln\left|\tan(x)+\sec(x)\right|+C.
\end{align*}

\noindent En conclusi\'on, $\displaystyle \int \sec(x) \,dx= \ln\left|\tan(x)+\sec(x)\right|+C$.
\end{proof}


%\begin{Ejemplo}
%Hallar $\displaystyle\int\csc(x) \,dx$
%\end{Ejemplo}
%\solucion \\
% Realizando los pasos analogos al ejercicio anterior, debemos multiplicar y dividir
%$\cot(x)+\csc(x)$ en el integrando
%
%\begin{equation*}
%\int\csc(x)\,dx=\int\csc(x)\frac{\cot(x)+\csc(x)}{\cot(x)+\csc(x)}\,dx=\int\frac{\csc(x)\cot(x)+\csc^2(x)}{
%\cot(x)+\csc(x)}\,dx.
%\end{equation*}
%
% \noindent Sea $\red{u=\csc(x)+\cot(x)}$ \ entonces \ $\red{du=(-\csc(x)\cot(x)-\csc^2(x)})\,dx$.
%
%\begin{align*}
%\int \csc(x)\,dx&=\int\frac{\csc(x)\cot(x)+\csc^2(x)}{\cot(x)+\csc(x)}\,dx \\
%&=-\int\frac{du}{u}
%\\&=-\ln\left|\csc(x)+\cot(x)\right|+C.
%\end{align*}

%%%%%%%%%%%%%%%%%%%%%%%%%%%%%%%%%%%%%%%%%%%%%%%%%%%%%%%%%%%%%%%%%%%%%%%%%%%%%%%%%%%%%%%%%%%%%%%%%%%%%%%%%%%%%%%%%%%%%%%%%%%%%%%
\vspace*{0.5cm}
El resto de integrales resulta de forma similar a las anteriores. 

\begin{teorema}{}{}
\begin{enumerate}
   \item $\displaystyle \int  \tan(u)\,du=-\ln \left|\cos (u)\right|+ C$
  \item $\displaystyle \int   \sec(u)\,du=\ln\left| \sec (u)+\tan(u)\right|+ C$
   \item $\displaystyle \int  \cot(u)\,du=\ln \left| \sen(u) \right|+ C$
  \item $\displaystyle \int  \csc(u)\,du=-\ln \left| \csc(u) +\cot(u) \right| +   C$
 \end{enumerate}
 \end{teorema}

%%%%%%%%%%%%%%%%%%%% 

%\begin{Ejemplo} Calcular $\displaystyle\int \sqrt{1+\tan^{2}(x)}\,dx$
%\end{Ejemplo}
% \solucion
%\begin{align*}
%\int \sqrt{1+\tan^{2}(x)}\,dx&=\int\sqrt{\sec^{2}(x)}\,dx && {\blue \mbox{por la
%identidad: }1+\tan^{2}(x)=\sec^{2}(x)}    \\
%&=\int\sec(x)\,dx\\
%&=\ln\left|\sec(u)+\tan(u)\right|+C.
%\end{align*}

%\begin{Ejemplo} Calcular $\displaystyle\int \frac{4-3\sen 2x}{\cos 2x} \, dx $ \end{Ejemplo}
%\solucion \\ Sea $\red{u=2x}$ entonces $\red{du=2dx}$ de modo que $\red{\frac{du}{2}=dx}$. Entonces
%\begin{align*}
% \int \frac{4-3\sen(2x)}{\cos(2x)}\,dx&=\int \frac{4-3\sen(u)}{\cos(u)}\,\frac{du}{2}\\
% &=2\int\sec(u)\,du-\frac{3}{2}\int\tan(u)\,du \\
%  &=2\ln|\sec(u)+\tan(u)|+\frac{3}{2}\ln|\cos(u)|+C\\
%  &=2\ln|\sec(2x)+\tan(2x)|+\frac{3}{2}\ln|\cos(2x)|+C.
%\end{align*}

%\begin{Ejemplo}
%Hallar $\displaystyle\int\frac{\tan(\ln(x))}{x}\,dx$
%\end{Ejemplo}
%
%\solucion Considerando $\red{u=\ln(x)}$, tenemos que $\red{du=\frac{dx}{x}}$. Por lo tanto
%
%\begin{align*}
%\frac{\tan(\ln(x))}{x}\,dx&=\int \tan(u)\,du \\
% &=-\ln|\cos(u)|+C=-\ln|\cos(\ln (x))|+C.
%\end{align*}
\newpage
\begin{Ejemplo}
 Halla $\displaystyle\int \frac{\tan^2 (2x)}{\sec (2x)}\, dx$
\end{Ejemplo}
\begin{proof}
Se tiene que
 \begin{eqnarray*}
 \int \frac{\tan^2 (2x)}{\sec (2x)}\, dx &=& \int \frac{ \sec^2 (2x)-1}{\sec (2x)}\, dx\\
 &=& \int \sec   (2x) \, dx - \int \cos (2x) \, dx\\
 &=&  \frac{1}{2} \ln|\sec (2x) + \tan (2x)|- \frac{1}{2} \sen (2x) + C
 \end{eqnarray*}
\end{proof}


%%%%%%%%%%%%%%%%%%%%%% 



\begin{Ejemplo} Calcular $\displaystyle\int \frac{\sen^2 (x) - \cos^2 (x)}{\cos(x)}\, dx$
\end{Ejemplo}
\begin{proof}
 \begin{eqnarray*}
 \displaystyle\int \frac{\sen^2 (x) - \cos^2 (x)}{\cos(x)}\, dx &=& \int \frac{1- \cos^2(x) - \cos^2(x)}{\cos (x)}\,
dx
 \\
   &=& \displaystyle\int \frac{1- 2 \cos^2 (x)}{\cos (x)}\, dx \\
   &=& \displaystyle\int \sec (x) \, dx - 2 \int\cos (x) \, dx \\ \\
   &=& \displaystyle\ln|\sec (x) + \tan (x)| - 2 \sen (x) + C
 \end{eqnarray*}
\end{proof}

Para resolver la integral 
$\displaystyle\int sec^2 (x) \ tan(x) dx$, si usamos la sustituci\'on \\ $t = tan(x)$ o la sustituci\'on $u = sec(x)$ encontramos la primitiva de $f(x)= sec^2 (x) \ tan(x)$. 
\begin{center}
?`por qu\'e obtiene resultados correctos y diferentes? 
\end{center}


%\begin{Ejemplo} Calcular $\displaystyle\int  \frac{\cot (\ln(x))}{x} dx$
%\end{Ejemplo}
%\solucion  \\
%Realizamos las sustitucionnes $\red{u=\ln x}$ \ y \ $\red{du=\frac{dx}{x}}$ tenemos que
%\begin{align*}
%\displaystyle\int  \frac{\cot (\ln(x))}{x} \ dx &  =  \displaystyle\int \cot (u) \ du=  \ln (\sen (u)) +C  \\
%& =  \ln \vert \sen ( \ln \vert x \vert ) \vert +C
%\end{align*}
%
%\begin{Ejemplo} Calcular  $\displaystyle\int\sen^3\left(x+\frac{1}{x}\right)\cos\left(x+\frac{1}{x}\right)\left(\dfrac{x^2-1}{x^2}\right)dx$
%\end{Ejemplo}
%\solucion  \\
%Sea $\red{u=\sen\left(x+\frac{1}{x}\right)}$ \ y \ $\red{du=\cos\left(x+\frac{1}{x}\right) \left(\dfrac{x^2-1}{x^2}\right)dx}$.
%\begin{align*}
%\displaystyle\int u^3 \ du = \frac{u^4}{4}+C =\frac{1}{4}\sen^{4}\left(x+\frac{1}{x}\right)+C
%\end{align*}
%
%\begin{Ejemplo} Calcular las integrales, utilizando la sustituci\'on indicada.
%$$\displaystyle\int \frac{\cos (x)}{\sqrt{1+\sen  (x)}}\ dx, \qquad t=\sen (x)$$
%\end{Ejemplo}
% \solucion  \\
%Sea $\red{t=\sen(x)}$ \ y \ $\red{dt=\cos(x)}$. Entonces
%\begin{align*}
% \displaystyle\int \frac{\cos (x)}{\sqrt{1+\sen  (x)}}\ dx = \displaystyle\int \frac{dt}{\sqrt{1+t}}\ dx
%\end{align*}
%Ahora, sea $\red{u=1+t)}$ \ y \ $\red{du=dt}$.Por lo tanto,
%\begin{align*}
% \displaystyle\int \frac{\cos (x)}{\sqrt{1+\sen  (x)}}\ dx = 2 \sqrt{1+\sen(x)}+C
%\end{align*}

\newpage

%%%%%%%%%%%%%%%%%%%%%%%%%%%%%%%%%%%%%%%%%%%%%%%%
\vspace*{1cm}
%\begin{changemargin}{-2.5cm}{0cm}
%\begin{ejer}~ \\ \\
%\begin{small}
\problemas{
\noindent Calcule la integral indefinida, usando si es necesario una o varias sustituciones.
\begin{multicols}{2}
\begin{enumerate}
\item $\displaystyle\int \frac{x^2+2x}{\sqrt{x^3+3x^2+1}} dx$
\item $\displaystyle\int x(x^2+1)\sqrt{4-2x^2-x^4} dx$
\item $\displaystyle\int 2\frac{x^3}{(x^2+4)^{3/2}} dx$
\item $\displaystyle\int \frac{x^3}{\sqrt{1-2x^2}} dx$
\item $\displaystyle\int 3x \sqrt{x+6} dx $
\item $\displaystyle\int 5x^2 \sqrt{1-x} dx $
\item $\displaystyle\int -\frac{x^2-1} {\sqrt{2x-1}} dx $
\item $\displaystyle\int \frac{2x+1} {\sqrt{x+4}} dx $
\item $\displaystyle\int \frac{x} {(x+1)-\sqrt{x+1}} dx $
%\item $\displaystyle\int x\sqrt[3]{x+4} dx $
\item  $\displaystyle\int x^2\sqrt{3-2x}  dx $
\item  $\displaystyle\int \frac{1}{x^2}\sqrt{1+\frac{1}{3x}}  dx $

\item  $\displaystyle\int \left(x+ \frac{1}{x}\right)^{3/2}\left(\frac{x^2-1}{x^2}\right) dx $

\item $\displaystyle\int \frac{(x+1)^2}{(\frac{x^3}{3}+x^2+x+5)^4} dx $
\item $\displaystyle\int (x^2+1)^{-3/2} dx $
\item $\displaystyle\int \frac{x}{\sqrt{\sqrt{(1+x^2)^3}+x^2+1}} dx $
\item $\displaystyle\int \frac{(x^2+1-2x)^{2/5}}{1-x} dx $
\item $\displaystyle\int \frac{x}{(x+1)-\sqrt{x+1}} dx $
%\item $\displaystyle\int \frac{x-1}{x-\sqrt{x}} dx $
\end{enumerate}
\end{multicols} 
%%\begin{multicols}{2}
%\noindent Calcule la integral indefinida, usando si es necesario una o varias sustituciones.
%\begin{multicols}{2}
%\begin{enumerate}
%\item $\displaystyle\int\frac{x^2+2x}{\sqrt{x^3+3x^2+1}} dx$ \\
%\item $\displaystyle\int\frac{x^3}{(x^2+4)^{3/2}} \ dx$\\
%\item $\displaystyle\int \frac{x^3}{\sqrt{1-2x^2}} \ dx$\\
%\item $\displaystyle\int x\sqrt{x+6} \ dx$\\
%\item $\displaystyle\int x^2\sqrt{1-x} \ dx$\\
%\item $\displaystyle\int\frac{x-1}{x-\sqrt{x}} \ dx$\\
%\item $\displaystyle\int \frac{x^2-1}{\sqrt{2x-1}} \ dx$\\
%\item $\displaystyle\int \frac{2x+1}{\sqrt{x+4}}\ dx$\\
%\item $\displaystyle\int \frac{x\ dx}{(x+1)-\sqrt{x+1}} $ \\
%\item $\displaystyle\int  \frac{\sin(\sqrt{x+1})}{\sqrt{x+1}} \,dx$ %%% ojo
%
%\item $\displaystyle\int (x+4)^{1/3} \ dx$\\
%\item $\displaystyle\int\frac{(1+\ln x)^2}{2x}dx$
%\item $\displaystyle\int x^2\sqrt{3-2x}dx$\\
%\item $\displaystyle\int\frac{1}{x^2}\sqrt{1+\frac{1}{3x}}dx$\\
%\item $\displaystyle\int sen^{2}(x)cos(x)dx$\\
%\item  $\displaystyle\int\frac{\sen (x)}{\cos^3 (x)}dx$ \\
%\item $\displaystyle\int\frac{\csc^2 (x)}{\cot^3 (x)}dx$\\
%\item  $\displaystyle\int \sec^2 (x)\sqrt{\tan (x)}dx$\\
%\item  $\displaystyle\int\sen x\sqrt{1-\cos xdx}$\\
%\item $\displaystyle\int \cos^2 (x) \sin (x) \,dx $  % ojo
%\end{enumerate}
%\end{multicols}


%\noindent Calcular las integrales, utilizando la sustituci\'on indicada.
%\begin{enumerate}
%\item[21] $\displaystyle\int \frac{dx}{x\sqrt{x^2-1}}dx, \qquad x=\frac{1}{t}$\\
%\item[22] $\displaystyle\int \frac{dx}{e^{x}+1}dx, \qquad x=-\ln (t)$ \\
%\item[23] $\displaystyle\int  x(5x^{2}-3)^{7}dx, \qquad 5x^{2}-3=t$ \\
%\item[24] $\displaystyle\int \frac{x \ dx}{x+1}\ dx, \qquad t=\sqrt{x+1}$
%\end{enumerate}

%\newpage

Calcule la integral indefinida de las siguientes funciones trigonom\'etricas.
\begin{multicols}{2}
\begin{enumerate}
\setcounter{enumi}{18}
\item $\displaystyle\int \frac{\sin (\sqrt{x})} {\sqrt{x}} dx $
\item $\displaystyle\int \frac{\sin (x)} {\cos^3 (x)  } dx $
\item $\displaystyle\int \frac{\cos (x)} {\sin^{5/2} (x)  } dx $
\item $\displaystyle\int \frac{\csc^2 (x)} {\cot^3 (x)  } dx $
\item $\displaystyle\int \frac{\sin (\frac{1}{x}) \cos (\frac{1}{x}) }{x^2} dx $
\item $\displaystyle\int \sec ^2 (x)\sqrt{\tan (x)} dx $
\item  $\displaystyle\int \frac{\sin\sqrt{x}}{\sqrt{x}} dx $
\item $\displaystyle\int \sin (x)\sqrt{1-\cos (x)} dx $
\item $\displaystyle\int \sin^3 (x)   dx $
\item $\displaystyle\int \cos^3 (x)   dx $
\item  $\displaystyle\int \sin^3(x+\frac{1}{x})\cos(x+\frac{1}{x})(\frac{x^2-1}{x^2}) dx $
\item  $\displaystyle\int \frac{\tan^3\sqrt{x}\sec^2 \sqrt{x}}{\sqrt{x}} dx $
\item  $\displaystyle\int \frac{\sin^4(1+\sqrt{x})\cos (1+\sqrt{x})}{\sqrt{x}} dx $
\item  $\displaystyle\int \tan ^2 xdx $
\item  $\displaystyle\int \cot ^2 xdx $
\item  $\displaystyle\int [\tan (x/3)+ \cot (x/3)]^2  dx $
\end{enumerate}
\end{multicols} 

Calcule la integral indefinida, haciendo si es necesario una sustitución.
\begin{multicols}{2}
\begin{enumerate}
\setcounter{enumi}{34}
\item $\displaystyle\int \frac{1} {2x+7  } dx $
\item $\displaystyle\int \frac{\cos x} {2+\sin x  } dx $
\item $\displaystyle\int \frac{1} {x} \ln^3 x dx $
\item $\displaystyle\int \frac{(1+\ln x)^2} {2x  } dx $
\item $\displaystyle\int \frac{2-3\sin2x}{\cos 2x} dx$
\item $\displaystyle\int \frac{2x^3}{x^2 -4} dx$
\item $\displaystyle\int \frac{x^3}{\sqrt{1-2x^2}} dx$
\item $\displaystyle\int \frac{1}{x\ln x} dx$
\item $\displaystyle\int \frac{1}{\sqrt{x}(1+\sqrt{ x})} dx$
\item $\displaystyle\int \frac{2\ln x+ 1}{x[\ln^2 x+\ln x]} dx$
\item $\displaystyle\int \frac{2+\ln^2 x}{x[1-\ln x]} dx$
\item $\displaystyle\int \frac{\tan(\ln x)}{x} dx$
\item $\displaystyle\int 2 ^{x\ln x}(\ln x+1)dx$
\item $\displaystyle\int \frac{\tan ^22x}{\sec 2x} dx$
\item $\displaystyle\int \frac{\sec^2x}{\tan x} dx$
\item $\displaystyle\int \frac{\sin^2x-\cos^2x}{\cos x} dx$
\item $\displaystyle\int \frac{2-3\sin2x}{\cos 2x} dx$
\item $\displaystyle\int \frac{\sin 3x}{\cos 3x-1} dx$
\item $\displaystyle\int (\tan 2x -\sec 2x) dx$
\item $\displaystyle\int \frac{1}{\cos4x} dx$
\item $\displaystyle\int \frac{1}{2\sin(3x+1)} dx$
\item $\displaystyle\int \frac{x^4-5x^2+3x-4} {x  } dx $
\item $\displaystyle\int \frac{x^4-5x^2+3x-4} {x+1} dx $
\item $\displaystyle\int \frac{x^3} {x-2} dx $
\item $\displaystyle\int \frac{3x^2-5x+1} {x-4} dx $
%\item $\int \frac{1} {\sqrt[3]{x^2}-\sqrt{x}} dx $\;\; Sugerencia: Haga $x=z^6$.
\end{enumerate}
\end{multicols} 

% \noindent Calcule la integral indefinida, haciendo uso de los resultados sobre la funci\'on logar\'itimica.
%\begin{multicols}{2}
%\begin{enumerate}
%    \item[25] $\displaystyle\int\frac{2-3\sen 2x}{\cos 2x}$ \\
%    \item[26] $\displaystyle\int\frac{2x^3}{x^2-4}dx$ \\
%    \item[27] $\displaystyle\int\frac{x^3}{\sqrt{1-2x^2}}dx$ \\
%    \item[28] $\displaystyle\int\frac{1}{x\ln x}dx$ \\
%    \item[29] $\displaystyle\int\frac{1}{\sqrt{x}(1+\sqrt{x})}dx$ \\
%\columnbreak
%\item[30] $\displaystyle\int\frac{2+\ln^2 x}{x[1-\ln x]}dx$ \\
%    \item[31] $\displaystyle\int\frac{\tan(\ln x)}{x}dx$ \\
%    \item[32] $\displaystyle\int 2^{x\ln x}(\ln x+1)dx$ \\
%    \item[33] $\displaystyle\int\frac{\tan^2 2x}{\sec 2x}dx$ \\
%    \item[34] $\displaystyle\int\frac{\sec^2 x}{\tan x}dx$ \\
%\end{enumerate}
%\end{multicols}



%%\item  Calcular las siguientes integrales
%\begin{enumerate}
% \begin{multicols}{2}
%    \item $\displaystyle\int \dfrac{x+1}{\sqrt{x^2 +2x+3}} \,dx $

% \item $\displaystyle\int  x^2 \sqrt{x+1}\,dx$
% \item $\displaystyle\int x^2(x^3+5)^4 \,dx$
% \columnbreak
%  \item $\displaystyle\int x \sin(x^2) \, dx  $
% \item $\displaystyle\int (8x^2+1)^2(16x)dx$
% \item $\displaystyle\int 4x^3 \sin(x^4) \, dx $
% \item  $  \displaystyle\int \dfrac{\sqrt{1+\sqrt{x}}}{\sqrt{x}} \quad d x $
%\item $\displaystyle\int \dfrac{ 12x^3 \sqrt{ \left( ln(x^4+1) \right)^3  }}{x^4 +1} \, dx $
%  \end{multicols}
% \end{enumerate}

% \item En los  aplicar el método de sustitución para calcular las integrales.
%\begin{enumerate}
%\begin{multicols}{2}
%\item $\int \sqrt{2x+1}\, dx$
%\item $\int x\sqrt{1+3x} \, dx$
%\item $\int x^2\sqrt{x+1} \, dx$
%\item $\int\limits_{-2/3}^{1/3} \frac{x \, dx}{\sqrt{2-3x}}$
%\item $\int \frac{(x+1) \, dx}{(x^2 + 2x +2)^3}$
%\item $\int \sen^3 x \, dx$
%\item $\int z(z-1)^{1/3}\, dz$
%\item $\int\frac{\cos x \, dx}{\sen^3 x}$
%\item $\int\limits_0^{\pi/4} \cos 2x\sqrt{4-\sen 2x} \, dx$
%\item $\int\frac{\sen x \, dx}{(3 + \cos x)^2}$
%\columnbreak
%\item $\int\frac{\sen x \, dx}{\sqrt{\cos^3 x}}$
%\item $\int\limits_3^8\frac{\sen \sqrt{x+1} \, dx}{\sqrt{x+1}}$
%\item $\int x^{n-1}\sen x^n\, dx$, $n\ne 0$
%\item $\int\frac{x^5 \, dx}{\sqrt{1-x^6}}$
%\item $\int t(1+t)^{1/4}\, dt$
%\item $\int (x^2+1)^{-3/2}\, dx$
%\item $\int x^2(8x^3+27)^{2/3}\, dx$
%\item $\int\frac{(\sen x + \cos x) \, dx}{(\sen x - \cos x)^{1/3}}$
%\item $\int\frac{x \, dx}{\sqrt{1+ x^2 + \sqrt{(1+x^2)^3}}}$
%\item  $\int\frac{(x^2+ 1 - 2x)^{1/5} \, dx}{1-x}$
%\end{multicols}
%\end{enumerate}
%\end{multicols}
%\end{small}
%\end{ejer}
%\end{changemargin}
                      }
\newpage
\restoregeometry %restaura los margenes de la p\'agina
