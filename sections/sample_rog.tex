%
% Sample SBC book chapter
%
% This is a public-domain file.
%
% Charset: ISO8859-1 (latin-1) áéíóúç
%
\documentclass{SBCbookchapter}
\usepackage[utf8]{inputenc}
\usepackage[T1]{fontenc}
\usepackage[brazil,english]{babel}
\usepackage{graphicx}
\usepackage{verbatim}
%\author{Darling Vasquez Cassis}
\title{Aplicaciones de la integral}

\begin{document}

\maketitle


\parbox{70mm}{¿Que vas a aprender?\\A usar la integral definida para determinar el área, dada una región plana limitada por dos o más curvas. Plantear una integral que permita determinar el volumen de un sólido de revolución, introduciendo elementos diferenciales convenientes, usando métodos como el de discos, arandelas o capas.
Usar la integral definida para calcular la longitud de una curva plana, que es la gráfica de una función con derivada continua en un intervalo cerrado.}
\hfill
\parbox{50mm}{Subtemas:
\begin{enumerate}
    \item Área bajo una curva.
    \item Área de una región entre dos curvas.
    \item Volumen: Métodos de los discos, arandelas y  capas.
    \item Longitud de arco.
    \item Integrales impropias.
\end{enumerate}}

\textbf{Mapa conceptual del capítulo:}
\begin{center}
    \includegraphics[scale=0.6]{mapa.png}
\end{center}
\begin{center}
    \textbf{Resumen}
\end{center}
En este capítulo se evoca el concepto de antiderivada, estudiado anterirormente, para dar solución al problema de calcular el área de una región plana, utilizando el teorema fundamendal del cálculo.
Se estudian las herramientas para plantear una integral que permita determinar el volumen de un sólido en revolución, introduciendo elementos diferenciales convenientes, usando métodos como el de los discos, las arandelas y las capas, también llamadas cascarones cilíndricos.
Además se usa la integral definida para calcular la longitud de una curva plana, que es la gráfica de una función con derivada continua en un intervalo cerrado.

\section{Area bajo una curva (Breve repaso)}

\usepackage{multirow}

\begin{table}[htb]
\centering
\begin{tabular}{|l|l|l|l|}
\hline
& \multicolumn{3}{c|}{Europa} \\
\cline{2-3}
& Ciudad & Río & Símbolo\\
\hline \hline
\multirow{3}{1cm}{España} & Madrid & Manzanares & Cibeles\\ \cline{2-4}
& Sevilla & Guadalquivir & Giralda\\ \cline{2-4}
& Zaragoza & Ebro & Pilar\\ \cline{1-4}
Francia & París & Sena & Torre Eiffel\\ \cline{1-4}
\multirow{2}{1cm}{Italia} & Roma & Tíber & San Pedro\\ \cline{2-4}
& Milán & \multicolumn{1}{c|}{-} & Duomo\\ \cline{1-4}
\end{tabular}
\caption{Tabla muy bonita.}
\label{tabla:final}
\end{table}







\begin{tabular}{|c|c|c|c|c|c|}
\hline
\multicolumn{5}{|c|}{Aplicaciones de la Integral}\\
\hline
Aplicación & Problema & \multicolumn{3}{|c|}{Fórmula} \\
\hline \hline
Area bajo una curva & Hallar el área bajo $f(x)$, entre  & \int_{a}^{b}  & \int_{a}^{b}  & \\

\cline{1-5}
Area entre curvas & ocho & \multicolumn{3}{|c|}{nueve} \\

\cline{1-5}
Volumen por discos & once & doce & trece & \\

\cline{1-5}
Volumen por arandelas & dos & tres & cuatro  & \\
\cline{1-5}
Volumen por capas & Hallar el volumen generado al girar  & tres & cinco  & \\
\cline{1-5}
Longitud de arco & Encuentrar la longitud de $f(x)$ en $[a,b]$ & seis   & seis  &\\
\cline{1-5}
Integral Impropia & Calcular la integral impropia & dos & tres & siete \\
\hline
\end{tabular}


\end{document}