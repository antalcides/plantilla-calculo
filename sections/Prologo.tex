 
\chapter*{Pr\'ologo}

\section*{Autores}
%\begin{multicols}{2}
\justify
\noindent \textbf{Javier de la Cruz}\\
Estudios posdoctorales en la Universidad de Z\'urich, Suiza (2016-2017). Doctor en Matem\'aticas de la Universidad Magdeburg, Alemania (2012). M.Sc. en Matem\'aticas de la Universidad Nacional de Medell\'in, Colombia (2007). Especialista en Matemáticas de la Universidad del Norte, Barranquilla, Colombia. (2004). Licenciado en Matem\'aticas y F\'isica de la Universidad del Atl\'antico (2001). Profesor Tiempo Completo del Departamento de Matem\'aticas y Estad\'isticas de la Universidad del Norte (2012-). \\


\noindent \textbf{Carlos de Oro} \\
Mag\'ister en Matem\'aticas y Estadist\'ica de la Universidad del Norte. Matem\'atico de la Universidad de C\'ordoba. Profesor Tiempo Completo del Departamento de Matem\'aticas y Estad\'isticas de la Universidad del Norte. \\

\noindent \textbf{Stiven D\'iaz} \\
Mag\'ister en Matem\'aticas de la Universidad del Norte, Colombia (2015). Licenciado en Matem\'aticas de la Universidad del Atl\'antico, Colombia (2015). Profesor del Departamento de Matem\'aticas y Estad\'isticas de la Universidad del Norte. \\
%\columnbreak

\noindent \textbf{Rogelio Grau} \\
Doctor en Matem\'aticas de la Universidad de Sao Paulo, Brasil. M.Sc. en Matem\'aticas de la Universidad del Norte. Matem\'atico de la Universidad del Atl\'antico. Profesor Tiempo Completo del Departamento de Matem\'aticas y Estad\'isticas de la Universidad del Norte. \\

\noindent \textbf{Darling Varsquez} \\
Mag\'ister en Matem\'aticas de la Universidad del Norte. Matem\'atica de la Universidad del Atl\'antico. Profesor del Departamento de Matem\'aticas y Estad\'isticas de la Universidad del Norte. \\
%\end{multicols}

%\newpage
%---------------------------------------
%%---------------------------------------
\section*{Caracter\'isticas del libro}
El presente libro surge como el deseo de los autores por organizar y compartir los apuntes de clase elaborados en su labor docente a lo largo de los últimos años al frente de la asignatura de cálculo integral, en los programas de ingeniería, geología y matemáticas de la Universidad del Norte. Por ello, inicialmente, el objetivo central de este proyecto es suministrar un texto guía que se ajuste a la parcelación de dichos programas, pero que con el tiempo su uso pueda ser extendido a otras instituciones. En ese orden de ideas, hemos realizado una selección adecuada del contenido programático, presentando solamente los temas acordes al pensum de los programas de pregrado antes mencionados.

%Estos apuntes surgen como el deseo de los autores por condensar y compartir  las notas de clases elaboradas a lo largo de los últimos años al frente de la enseñanza del cálculo integral  en los programas de ingeniería, geología y matemáticas de la Universidad del Norte. Por ello, inicialmente, el objetivo central de este proyecto es suministrar un texto guía que se ajuste a la parcelación de dichos programas, pero que con el tiempo su uso pueda ser extendido a otras universidades. En ese orden de ideas hemos realizado una selección adecuada del contenido programático, presentando solamente los temas acordes al pensum de los programas de pregrados antes mencionados.

En el texto los conceptos son presentados intentando que estos sean lo más asequible posible para el estudiante, pero sin alejarnos de la formalidad y rigurosidad matemática en las definiciones y resultados. Asimismo, con respecto a las demostraciones de los teoremas y propiedades estudiadas en este texto, los autores desarrollan sólo las que consideran importantes y que pueden contribuir al alcance de los objetivos del curso. Sin embargo, se hace una continua invitación al lector a establecer o consultar las pruebas restantes.

Con el fin de hacer un texto didáctico que acompañe el proceso de ense\~nanza-aprendizaje, estos apuntes cuentan con un gran número de problemas resueltos de forma detallada, que lo convierten en un punto medio entre problemario y libro tradicional. Además pretendemos, mediante ejercicios cuidadosamente escogidos, la construcción y aplicación de los conceptos abordados y un aprendizaje autodidacta.

Algunas componentes que hacen parte de la estrategia metodológica y de la estructura del texto son:
\begin{itemize}

 \item Los ejercicios propuestos  que gradualmente se hacen m\'as complejos, hasta que el estudiante alcance un nivel avanzado de destreza.
\item	Relaci\'on de las matem\'aticas con otras \'areas del conocimiento.
\item	Variedad de actividades que incluyen problemas y ejercicios
\item	\'Enfasis en la compresi\'on y el uso del lenguaje matem\'atico, para comunicar ideas y generar pensamiento abstracto.
\item	Respuestas a los ejercicios impares planteados. 
\end{itemize}

Es importante aclarar que no suministramos las respuestas a los ejercicios pares, para generar confianza en el estudiante al momento de su realización. 

En cuanto a la distribución del contenido, éste ha sido organizada en cuatro unidades generales. La primera unidad, llamada Integración, inicia con el concepto de antiderivada, como operaci\'on inversa a la derivaci\'on y se centra en la presentación de la integral
indefinida de funciones algebraicas y trascendentes, haciendo énfasis en el m\'etodo de sustituci\'on. Además se aborda el problema de la determinación del área de la región bajo una curva, el concepto de integral definida y los dos teoremas fundamentales del c\'alculo. Posteriormente, en la unidad M\'etodos de integraci\'on, son abordados el m\'etodo de integraci\'on por partes, el m\'etodo de integraci\'on por sustitución trigonométrica y el m\'etodo de integraci\'on de fracciones parciales. En la tercera Unidad, denominado Aplicaciones, usaremos la integral definida para determinar el área de dada una regi\'on plana limitada por dos o m\'as curvas y el volumen de un s\'olido de revoluci\'n, introduciendo elementos diferenciales convenientes. Tambi\'en presentaremos las herramientas para calcular la longitud de arco una curva plana. Finalmente, en la cuarta unidad, denominada Series, se estudian las series reales y los principales criterios de convergencia.


\section*{Objetivo del libro}

Brindar a los estudiantes de los programas de ingeniería, geología y matemáticas, un documento complementario en su proceso de aprendizaje del c\'alculo integral y su conexi\'on con el ejerecicio profesional.

\section*{Competencias}

El texto pretende contribuir al desarrollo de las siguientes competencias:

\begin{enumerate}

\item[$\bullet$] Comprende los conceptos de primitiva, integral definida de una funci\'on y los dos teoremas fundamentales del c\'alculo.

\item[$\bullet$] Aplicar los conocimientos operativos necesarios para el c\'alculo de integrales de funciones polin\'omicas, racionales, algebraicas trascendentes.

\item[$\bullet$] Utilizar las t\'ecnicas de integraci\'on para modelar explicaciones o soluciones a problemas espec\'ificos del \'area profesional en el que se encuentran formando.

\item[$\bullet$] Comprende  las sumas infinitas como una sucesión y establecer a partir de criterios básicos su convergencia o divergencia.

\end{enumerate}


%{\red Objetivos}
%\begin{itemize}
% \item Usar integraci\'on por partes para calcular integrales indefinidas y definida
% \item Hallar las integrales usando el método de tabulaci\'on.
% \item Usar sustituciones trigonom\'etricas para hallar integrales.
% \item Aplicar de forma acertada una descomposición en fracciones simples o parciales
% \item Usar descomposici\'on de fracciones simples con los factores lineales o cuadr\'aticos para calcular
%las integrales de funciones racionales
% \end{itemize}

%\section*{Complemento del texto}
%
%Los autores en funci\'on de su quehacer como docente de la Universidad del Norte colocan a la disposici\'on de sus estudiantes el siguiente sitio web donde encontraran diferentes materiales que pueden utilizarce como complemento al libro;  www.uninorte.edu.co/web/departamento-de-matematicas-y-estadistica/calculo-2









